\documentclass[10pt, xcolor=x11names,compress]{beamer}
\usepackage{tabulary}
\usepackage{booktabs}
\usepackage{float}
\usepackage{graphicx}
\usepackage{mwe}% for example pictures
\usepackage{siunitx}
\usepackage{hyperref}
\usepackage{listings}

\usecolortheme{spruce}
\useoutertheme{infolines}
\usefonttheme[onlymath]{serif}
\setbeamertemplate{headline}[default]
\setbeamertemplate{navigation symbols}{}
\mode<beamer>{\setbeamertemplate{blocks}[rounded][shadow=true]}
\setbeamercovered{transparent}
\setbeamercolor{block body}{use=structure, fg=white, bg=black!20}
\setbeamercolor{itemize item}{fg=black}
\setbeamercolor{itemize subitem}{fg=gray} 
\setbeamercolor{itemize subsubitem}{fg=black!20} 
\makeatletter\setbeamertemplate{footline}
{  
\leavevmode%  
\hbox{%  
\begin{beamercolorbox}[wd=.333333\paperwidth,ht=2.25ex,dp=1ex,center]{author in head/foot}%    
\usebeamerfont{author in head/foot}
\insertshortauthor%~~\beamer@ifempty{\insertshortinstitute}{}
 \end{beamercolorbox}%  
 \begin{beamercolorbox}[wd=.333333\paperwidth,ht=2.25ex,dp=1ex,center]{institute in head/foot}%    
 \usebeamerfont{title in head/foot}\insertinstitute  
 \end{beamercolorbox}%  
 \begin{beamercolorbox}[wd=.333333\paperwidth,ht=2.25ex,dp=1ex,right]{date in head/foot}%    
 \usebeamerfont{date in head/foot}\insertshortdate{}\hspace*{2em}    
 \insertframenumber{} / \inserttotalframenumber\hspace*{2ex}   
 \end{beamercolorbox}}%  
 \vskip0pt%
 }
 \makeatother 
 \useoutertheme[footline=empty, subsection=false]{miniframes}
 \usepackage{multicol}  
 \author{Miro Kurka}
 \title{Garbage Collection in JVM}

 \institute{Pavol Jozef Safarik University}\date{Seminar series \\ 19. 10. 2023} 
 \begin{document}
 \begin{frame}
 \titlepage
 \end{frame}

\section{Introduction}
\begin{frame}[label=Background]{Background}
\begin{itemize}
\item Why GC?\\
   \begin{itemize}
    \item This seminar is about \texttt{automata} and ML 
    \item Read a paper\footnote{link to paper}\
    \item Low latency programming
    \item Our University basic course is Java 
    \item Jobs in KE are usually Java or some dialects (Kotlin,Scala, etc...)
    \item This book\footnote{That means for all statements x about GC \contains {JVM (2012)}}
   \end{itemize}
\item Now we go back to the main content\\

\end{itemize}
\end{frame}

\section{Content}
\begin{frame}{What I want to present to you}
\begin{itemize}
    \item Memory layout
    \item Introduction to GC in Java 
    \item Serial GC 
    \item G1 GC 
    \item CMS 
\end{itemize}
\end{frame}

\section{Memory layout}
\begin{frame}[label=Rocky]{There you can also separate the page}
\begin{figure}
 \begin{columns}[c]
  \begin{column}{0.3\textwidth}
    \centering
    \includegraphics[width=1\textwidth]{Figure2.jpg}
  \end{column}
  \begin{column}{0.3\textwidth}
    \centering
    \includegraphics[width=1\textwidth]{Figure3.jpg}
  \end{column}
  \end{columns}
  \caption{Here I just using one caption for these 2 cute pictures but you can also using 2 captions  \parbox{\linewidth}{\small\textit{Data source: Rocky's daily look}}}
\end{figure}
Now I am going back to the previous slidesby clicking this: \hyperlink{Background}{go back}
\end{frame}
 
\begin{frame}{Introduction to GC in Java}

\begin{itemize}
 \item Java offers automatic memory management (a blessing and a curse)
 \item GC is about finding objects no longer in use and reclaiming their memory
 \item Simple reference counting is not enough, JVM searches the heap
 \item Defragmentation  
\end{itemize}
\end{frame}

\begin{frame}{Overview of GC algorithms}

  \begin{itemize}
   \item All GC algorithms in Java split heap into different generations
   
  \end{itemize}

\end{frame}
  
\section{Data}
\begin{frame}{Data Source}
His father's statistics
   
\end{frame}

\section{Results}

\section{Conclusion}
\begin{frame}{Conclusion}
\begin{itemize}
    \item  abc
    \item  more info is in handbook of GC, new edition this year
\end{itemize}
    
\end{frame}


\begin{frame}
 \begin{center}
		{Thank you!}\\
		\bigskip\bigskip % Vertical whitespace
		
		{\LARGE Rocky@cutedog.edu}
		
	\end{center}
\end{frame}

\end{document}
